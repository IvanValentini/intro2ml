\chapter{Gradient descent}

\section{Model based machine learning}
Nel model based machine learning si sceglie un modello definito da un insieme di parametri.
In particolare si nota come:
\begin{multicols}{2}
	\begin{itemize}
		\item Per i decision trees i parametri sono la struttura dell'albero, quali features ogni nodo divide e le predizioni delle foglie.
		\item Per il perceptron i parametri sono i pesi e il valore di $b$.
	\end{itemize}
\end{multicols}
Dopo aver scelto il modello si deve scegliere un criterio da ottimizzare o la funzione obiettivo come per esempio il training error.
Infine si sviluppa un algoritmo di learning che deve cercare di minimizzare il criterio, spesso in maniera euristica.

	\subsection{Modelli lineari}
	Nei modelli lineari il modello \`e:
	$$0=b+\sum\limits_{j=1}^mw_jf_j$$
	Si deve scegliere il criterio da ottimizzare.

		\subsubsection{Notazioni}

			\paragraph{Funzione indicatrice}
			Una funzione indicatrice trasforma valori di \emph{Vero} e \emph{Falso} in numeri e conte.
			$$1[x]=\begin{cases}1&\mathbf{if} x = True\\
										  	0&\textbf{if} x = False
				 	\end{cases}$$

			\paragraph{Dot-product}
			Utilizzando una notazione vettoriale si rappresenta un esempio $f_1,\dots,f_m$ come un vettore singolo $\overrightarrow{x}$ in cui $j$ indicizza la feature e $i$ indicizza un dataset di esempi.
			Si possono rappresentare anche i pesi $w_1,\dots,w_m$ come un vettore $\overrightarrow{w}$.
			Il dot-product tra due vettori $a$ e $b$ viene definito come:
			$$a\cdot b = \sum\limits_{j=1}^ma_jb_j$$

		\subsubsection{Funzione obiettivo}
		Il criterio da ottimizzare o funzione obiettivo pu\`o essere:
		$$\sum\limits_{i=1}^n1[y_i(w\cdot x_i+b)\le 0]$$
		Si devono pertanto trovare $w$ e $b$ tali che minimizzano questa funzione, ovvero:
		$$argmin_{w,b}\sum\limits_{i=1}^n1[y_i(w\cdot x_i+b)\le 0]$$

\section{Loss functions}

	\subsection{Loss $0/1$}
	Una funzione di loss $0/1$ \`e una funzione nella forma:
	$$\sum\limits_{i=1}^n1[y_i*w\cdot x_i+b)\le 0]$$
	Dove tra le quadre si trova se la predizione e la label sono d'accordo, con vero se non lo fanno e tra le tonde la distanza dall'iperpiano, di cui il segno \`e la predizione.
	Questa funzione ritorna il numero di sbagli.

		\subsubsection{Minimizzare la loss $0/1$}
		Per minimizzare una funzione $0/1$ si deve, ogni volta cambiare un valore di $w$ in modo che l'esempio \`e corretto o scorretto la perdita aumenta o diminuisce.
		Si nota come a ogni feature aggiunta si aggiunge una nuova dimensione allo spazio.
		Il minimo si trova trovando $w$ e $b$ che minimizzano la perdita.
		Questo \`e un problema \emph{NP-hard}.
		Sue difficolt\`a comprendono il fatto che piccoli cambi in ogni $w$ possono portare a grandi cambi nella perdita in quanto il cambio non \`e continuo.
		Ci possono essere molti minimi locali.
		Ad ogni punto non si hanno informazioni che direzionano verso il minimo.
		Pertanto si nota come una loss function ideale sia continua e differenziabile in modo da avere un'indicazione verso la direzione di minimizzazione e un unico minimo.

	\subsection{Funzioni convesse}
	In una funzione convessa il segmento tra qualsiasi due punti della funzione si trova al di sopra della funzione.

	\subsection{Surrogate loss function}
	Per molte applicazioni si vuole minimizzare la loss $0/1$.
	Una surrogate loss function \`e una loss function che fornisce un limite superiore alla loss function attuale.
	Si vuole identificare un surrogato convesso della loss function in modo da facilitarne la minimizzazione.
	Chiave a una loss function \`e come verifica la differenza tra la label $y$ effettiva e la predizione $y'$.

		\subsubsection{Alcune surrogate loss function}
		\begin{multicols}{2}
			\begin{itemize}
				\item \emph{$01$ loss}: $l(y, y')=1[yy'\le 0]$.
				\item \emph{Hinge} $l(y,y')=\max(0,1-yy')$.
				\item Exponential: $l(y,y')=\exp(iyy')$
				\item \emph{Squared loss}: $l(y,y')=(y-y')^2$.
			\end{itemize}
		\end{multicols}

\section{Gradient descent}
Il gradient descent \`e un modo per trovare il minimo di una funzione: le derivate parziali danno un slope o direzione dove muoversi in tale dimensione.
Questo approccio consiste di scegliere un punto di partenza e a ripetizione di: scegliere una dimensione e muoversi di una piccola quantit\`a verso il minimo utilizzando la derivata.

	\subsection{Spostamento in direzione della minimizzazione dell'errore}
	Il movimento in direzione della minimizzazione dell'errore \`e pertanto:
	$$w_j = w_j - \eta \dfrac{d}{dw_i}loss(w)$$
	Dove $\eta$ \`e il learning rate.

		\subsubsection{Calcolo dello spostamento per la loss function esponenziale}
		Si deve pertanto calcolare:
		\begin{align*}
			\dfrac{d}{dw_j}loss &=\dfrac{d}{dw_j}\sum\limits_{i=1}^n\exp(-y_i(w\cdot x_i + b))\\=
												 &=\sum\limits_{i=1}^n\dfrac{d}{dw_j}[-y_i(w\cdot x_i+b)]\exp(-y_i(w\cdot x_i+b))\\
		\end{align*}
		Si consideri pertanto ora:
		\begin{align*}
			\dfrac{d}{dw_j}[-y_i(w\cdot x_i + b)]&=-\dfrac{d}{dw_j}[-y_i(w\cdot x_i + b)]=\\
																		&=-\dfrac{d}{dw_j}y_i(w_1x_{i1}+\cdots+w_mx_{im}+b)=\\
																		&=-y_ix_{ji}
		\end{align*}
		Si nota pertanto come:
		$$\dfrac{d}{dw_j}loss=\sum\limits_{i=1}^n-y_ix_{ij}\exp(-y_i(w\cdot x_i+b))$$
		Si aggiorna pertanto $w_j$:
		$$w_j=w_j-\eta\sum\limits_{i=1}^n-y_ix_{ij}\exp(-y_i(w\cdot x_i+b))$$
		Questo viene fatto per ogni esempio $x_i$.

	\subsection{Learning algorithm del perceptron}
	Si nota pertanto come considerando il perceptron nell'ambito del gradient descent si aggiorna sempre il vettore dei pesi.
	\begin{algorithm}[H]
\DontPrintSemicolon
\SetKwComment{comment}{$\%$}{}
\SetKw{Int}{int}
\SetKw{To}{to}
\SetKw{IsNot}{is not}
\SetKw{Not}{not}
\SetKwData{Item}{item}
\SetKwFunction{Min}{min}
\SetKwFunction{Perceptron}{Perceptron}

\caption{\protect\Perceptron{}}

\Repeat{Convergence}{
	\ForEach{training example ($f_1, f_2,\dots,f_n, label$)}{
		\comment{Label = $\pm 1$}
		$prediction\ =\ b\ +\ \sum\limits_{i=1}^nw_if_i$\;
			\ForEach{$w_i$}{
				$w_j = w_j + \eta y_ix_{ij}\exp(-y_i(w\cdot x_i + b))$
				$b\ =\ b\ +\ label$\;
			}
		}
	}



\end{algorithm}

	Si noti come in questo caso $\eta$ rappresenta il learning rate, $y_i$ la label e $(w\cdot x_i + b))$ la predizione.
	Questi generano una costante $c$

	\subsection{Costante $\mathbf{c}$}
	Nella costante $c$ se label e predizione hanno lo stesso segno quando gli elementi predetti aumentano gli aggiornamenti diventano minori.
	Se invece sono diversi pi\`u diversi lo sono, maggiore l'aggiornamento.

	\subsection{Gradiente}
	Il gradiente \`e il vettore delle derivate parziali rispetto a tutte le coordinate dei pesi:
	$$\nabla L = \biggl[\dfrac{\delta L}{\delta w_1}\cdots\dfrac{\delta L}{\delta w_N}\biggr]$$
	Ogni derivata parziale misura quanto veloce la perdita cambia in una direzione.
	Quando il gradiente \`e zero la perdita non sta cambiando in nessuna direzione.
	
\begin{algorithm}
\DontPrintSemicolon
\SetKwComment{comment}{$\%$}{}
\SetKw{Int}{int}
\SetKw{To}{to}
\SetKw{IsNot}{is not}
\SetKw{Not}{not}
\SetKw{Return}{return}
\SetKwData{Item}{item}
\SetKwFunction{Min}{min}
\SetKwFunction{GradientDescent}i{GradientDescent}

\caption{\protect\GradientDescent{$\mathcal{F}$, $K$, $\eta_1$, $\dots$}}
	$z^{(0)}\ \rightarrow <0,0,\dots,0$\comment{Inizializza la variabile da ottimizzare}
	\For{$k = 1$ \To $K$}{
		$g^{(k)}\ \rightarrow\ \nabla_z\mathcal{F}|_{z^{(k-1)}}$\comment{Computa il gradiente nella posizione corrente}
		$z^{(k)} \rightarrow z^{(k-1)} - \eta^{(k)}g^{(k)}$\comment{scendi il gradiente}
	}
	\Return $z^{(k)}$\end{algorithm}

	Nei problemi in cui il problema di ottimizzazione \`e non convesso si trovano dei minimi locali, questi non permettono all'algoritmo di proseguire in quanto non distingue tra minimi locali e minimi globali.
	Un altro punto \`e un punto a sella, in cui certe direzioni curvano verso l'alto e altre verso il basso.
	In tali punti il gradiente \`e $0$ e l'algoritmo si blocca.
	Un modo per uscire da un punto \`a sella \`e spostarsi a lato un po' in modo da uscirne.
	Si nota come i punti a sella sono molti comuni in alte dimensioni.
	Il learning rate \`e molto importante in quanto permette di decidere la distanza coperta da uno spostamento determinando velocit\`a di avvicinamento al minimo e precisione dell'algoritmo.
