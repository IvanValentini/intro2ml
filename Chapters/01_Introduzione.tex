\chapter{Introduzione}

\section{Definizioni}
Si intende per machine learning lo studio di algoritmi che migliorano autonomamente attraverso l'esperienza.
\`E un campo dell'intelligenza artificiale.
Stravolge il paradigma convenzionale della programmazione: un algoritmo di machine learning infatti prende come input un insieme di dati e risultati in modo da produrre un programma che fornisce un risultato appropriato.
Coinvolge pertanto la scoperta automatica di regolarit\`a nei dati attraverso algoritmi in modo da poter compiere azioni basate su di essi.

\section{Processo}
Il machine learning permette ai computer di acquisire conoscenza attraverso algoritmi che inferiscono e imparano da dati.
Questa conoscenza viene rappresentata da un modello che pu\`o essere utilizzato su nuovi dati.

	\subsection{Il processo di apprendimento}
	Il processo di apprendimento in particolare coinvolge diversi passaggi:
	\begin{multicols}{2}
		\begin{itemize}
			\item Acquisizione dei dati dal mondo reale attraverso dispositivi di misurazione come sensori o database.
			\item Preprocessamento dei dati: filtraggio del rumore, estrazione delle feature e normalizzazione.
			\item Riduzione dimensionale: selezione e proiezione delle feature.
			\item Apprendimento del modello: classificazione, regressione, clustering e descrizione.
			\item Test del modello: cross-validation e bootstrap.
			\item Analisi dei risultati.
		\end{itemize}
	\end{multicols}

\section{Modello}
Un algoritmo di machine learning impara dall'esperienza $E$ in rispetto di una classe di compiti $T$ e di misurazione delle performance $P$, se la $P$ di $T$ aumenta con $E$.
Si nota pertanto come un compito di machine learning ben definito possiede una tripla:
$$<T, P, E>$$

\section{Deep learning}
Il deep learning \`e un sottoinsieme del machine learning che permette a modelli computazionali composti di multipli strati di imparare la rappresentazione di dati con multipli livelli di astrazione.
Si utilizza pertanto una rete neurale con diversi strati di nodi tra input e output.
Questa serie di strati tra input e output computa caratteristiche rilevanti automaticamente in una serie di passaggi.
Questi algoritmi sono resi possibili da:
\begin{multicols}{2}
	\begin{itemize}
		\item Enorme mole di dati disponibili.
		\item Aumento del potere computazionale.
		\item Aumento del numero di algoritmi di machine learning e della teoria sviluppata dai ricercatori.
		\item Aumento del supporto dall'industria.
	\end{itemize}
\end{multicols}
