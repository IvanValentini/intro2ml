\chapter{Neural Networks}

\section{Introduzione}
\`E stato visto come il perceptron pu\`o imparare un modello lineare utilizzando la funzione attivatrice e il peso degli input.
Funzioni attivatrici tradizionali sono non lineari come la sigmoide $h(x) = \frac{1}{1+e^{-x}}$ e la tangente iperbolica $h(x) = \frac{e^x-e^{-x}}{e^x+e^{-x}}$, mentre funzioni attivatrici pi\`u moderne sono la rectified linear unit \emph{ReLU}: $h(x) = \max(0,x)$ e la Leaky ReLU $h(x) = \max(\alpha x,x)$ con $\alpha$ costante piccola.

	\subsection{Perceptron}
	Il perceptron si pu\`o considerare come un neurone artificiale, una funzione non lineare parametrizzata con un range di output ristretto:
	$$\hat{y} = h\bigl(\sum\limits_i\theta_ix_i\bigr) = h(\theta^Tx)$$

		\subsubsection{Rosemblatt}
		Nel $1958$ Rosemblatt considera il perceptron come una macchina per la classificazione lineare: si imparano i pesi e si considera il bias: un peso per input, si moltiplicano i pesi con gli input rispettivi e si aggiunge il bias.
		Se il risultato \`e pi\`u grande di una threshold si ritorna $1$, altrimenti $0$.

	\subsection{Multi layer perceptron}
	Il perceptron presenta diverse limitazioni: non \`e in grado di risolvere problemi non linearmente separabili come quello dello \emph{XOR}.
	Per superare questo problema Minsky e Papert nel $1969$ sviluppano il multi-layer perceptron \emph{MLP}.
	Questi sono neuroni artificiali densamente connessi che realizzano composizioni di funzioni non lineari utilizzati per la classificazione e regressioni.
	Sono composti da un input layer, diversi hidden layer e un output layer.
	L'informazione viene propagata dall'input all'output senza cicli: l'\emph{MLP} \`e un directed acyclic graph \emph{DAG}.
	La computazione viene svolta dalla composizione di un numero di funzioni algebriche implementate dalle connessioni, pesi e biases dei layer di output e nascosti.
	Gli hidden layer computano delle rappresentazioni intermedie.

		\subsubsection{Single layer neural network}
		In una single layer neural network il primo layer $(1)$ riceve i dati dall'input e li passa a un layer nascosto $(2)$, che a sua volta li passa all'output: $\hat{y}_k$:
		\begin{itemize}
			\item[$(1)$] $z_j = \sum\limits_i\theta^{(1)}_{i,j}x_i + \theta_{0,j}^{(1)}$.
			\item[$(2)$] $\hat{y}_k = h\bigl(\sum\limits_i\theta^{(2)}_{i,j}h(z_i)+\theta^{(2)}_{0,j}i\bigr)$.
			\end{itemize}

	\subsection{First AI winter}
	Il first AI winter nasce in quanto si rende difficile fare del training su un \emph{MLP}: non ci si pu\`o applicare la regola del perceptron in quanto si aspetta di conoscere il target desiderato.
	Per gli hidden layer infatti \`e impossibile conoscere il target desiderato.

		\subsubsection{Backpropagation}
		Il problema del training del \emph{MLP} viene risolto attraverso la backpropagation nel $1986$, che permette pertanto il learning di \emph{MLP} per funzioni complicate.
		Algoritmi efficienti permettono di processare grandi training sets e permettono architetture complesse di neural networks.
		Tutt'oggi si trova al nucleo del training delle reti neurali.
		Il training pertanto consiste di tre passaggi:
		\begin{multicols}{2}
			\begin{itemize}
				\item Forward propagation: si sommano gli input, si produce attivazione e si fa feed-forward.
				\item Si stima l'errore.
				\item Si propaga all'indietro il segnale di errore e lo si usa per aggiornare i pesi.
			\end{itemize}
		\end{multicols}
		Il learning diventa un problema di ottimizzazione, in cui dati i training samples $T=\{(x_1,y_1),\dots,(x_N,y_N)\}$ si devono aggiustare tutti i pesi della rete $\Theta$ in modo che una funzione di costo sia minimizzata:
		$$\min_\Theta\sum\limits_i L(y_i, f(x_i,\Theta))$$
		Si deve pertanto scegliere la loss function, aggiornare i pesi di ogni layer con il gradient descent e usare la backpropagation del segnale di errore per computare il gradiente efficientemente.

		\subsubsection{CNN e LSTM}
		La backpropagation permette importanti sviluppi nel campo come le convolutional neural networks e le recurrent long-short term memory networks negli anni $90$.

	\subsection{Second AI winter}
	Si nota come queste reti neurali non possono sfruttare molti layer a causa di overfitting e del vanishing gradient: la moltiplicazione di piccoli numeri nel training causa a questi di diventare sempre pi\`u piccoli fino a renderli non significativi.
	Inoltre non si disponeva della capacit\`a computazionale necessaria e mancavano grandi dataset annotati.

		\subsubsection{SVM e metodi di kernel}
		A causa di questi problemi diventano popolari le Kernel machines come gli SVM in quanto riuscivano ad ottenere accuratezze simili alle reti neurali, possedevano meno euristiche e parametri e una prova di generalizzazione.

	\subsection{Deep learning revolution}
	Nel $2006$ si utilizza un nuovo modo per inizializzare i pesi: ogni layer viene trainato singolarmente attraverso unupervised training come contrastive divergence e i pesi si sistemano con un round di supervised learning.
	Nasce cos\`i alexnet, una rete CNN simile a LeNet ma trainata con due GPU e con migliorie tecniche come ReLU, dropout e data augmentation.

	\subsection{Caratteristiche del deep learning}
	Si nota come il deep learning, a differenza del machine learning tradizionale non richiede la creazione di feature a tavolino: la loro estrazione avviene infatti negli hidden layers.
	Si nota pertanto come una rete neurale \`e una composizione di moduli, funzioni connesse gerarchicamente con parametri $\Theta$.

\section{Feedforward networks}
Nelle reti feed forward la funzione $f$ \`e la composizione di multiple funzioni:
$$f(x) = f^{(n)}(\dots(f^{(1)}*x)\dots)$$
L'obiettivo \`e quello di approssimare una funzione sconosciuta ideale $f^*:\mathcal{X}\rightarrow\mathcal{Y}$, in cui il modello ideale \`e $y = f^*(x)$.
Per queste reti si definisce una mappatura parametrica $y = f(x,\theta)$ e si imparano i parametri per trovare una buona approssimazione di $f^*$ dai sample disponibili.
L'informazione fluisce dall'input, verso computazioni intermedie e si produce l'ouput.
La composizione delle funzioni pu\`o essere descritta come un \emph{DAG}, in cui la profondit\`a \`e il massimo $i$ nella catena di composizione delle funzioni.
Il layer finale viene chiamato layer di output.

	\subsection{Training}
	Per il training si deve ottimizzare $\theta$ per portare $f(x,\theta)$ il pi\`u vicino possibile a $f^*(x)$.
	Questa viene valutata a diverse istanze di $x$ di training data, che specifica solamente l'output dei layer finali.
	L'output dei layer intermedi non viene specificato, da cui il nome hidden layers.
	Questo \`e simile a progettare una macchina basata su gradient descent: il problema diventa non convesso e non c'\`e garanzia di convergenza.
	In particolare per applicare il gradient descent si deve specificare un modello, una funzione di costo e la rappresentazione dell'output.

	\subsection{Scelte di modellamento}
	Per modellare una feedforward network si devono scegliere:
	\begin{multicols}{2}
		\begin{itemize}
			\item Funzione di costo.
			\item Forma dell'output.
			\item Funzione di attivazione.
			\item Architettura.
			\item Ottimizzatore.
		\end{itemize}
	\end{multicols}

		\subsubsection{Funzione di costo}
		La funzione di costo dice quanto la rete si composta bene rispetto ai training data:
		$$\mathcal{L}(w) = distance(f_\theta(x), y)$$
		Calcola ovvero la discrepanza tra la predizione e la label effettiva,
		Si possono applicare loss gi\`a viste e tipicamente si converte l'output in probabilit\`a come attraverso softmax.

			\paragraph{Cross-entropy}
			La cross-entropy loss \`e la loss function pi\`u comune applicata con softmax:
			$$\mathcal{L}_i = -\sum\limits_ky_y\log(S(l_k)) = \log(S(l))$$

		\subsubsection{Output layers}
		Si nota come la scelta della loss function sia correlata alla scelta dell'unit\`a di output.

			\paragraph{Linear}
			Data una feature $h$ un unit\`a di un layer di output lineari d\`a:
			$$\hat{y} = W^Th+b$$
			Questi unit\`a non si saturano (mantiene il gradiente lontano da $0$), offrono poca difficolt\`a agli algoritmi basati sull'ottimizzazione del gradiente.
			Si nota come un output del modello vicino a $0$ \`e problematico.

			\paragraph{Softmax}
			Softmax permette di produrre probabilit\`a normalizzate nell'output layer, pertanto produce:
			$$S(l_i) = \frac{e^{l_i}}{\sum\limits_ke^{l_k}}$$

		\subsubsection{Hidden units}
		All'interno di una hidden unit si accetta un input $x$, si computa una trasformazione affine: $z = w^Tx+b$, si applica element-wise una funzione non lineare $h(z)$ e si ottiene l'output $h(z)$.
		La scelta da compiere si basa su quale $h$ utilizzare.

			\paragraph{Rectified linear units (RELU)}
			Una RELU presenta un gradiente di $0$ o $1$, facile da ottimizzare simile alle unit\`a linari.
			D\`a gradienti grandi e consistenti quando attiva, ma non \`e sempre differenziabile, risolvibile attraverso una derivata da un lato a $z=0$.
			Ne esistono diverse varianti che risolvono il problema che l'unit\`a muore quando il gradiente \`e $0$:
			\begin{multicols}{2}
				\begin{itemize}
					\item ReLu: $y_i = 0$ per $x < 0$ o $y_i = x_i$ per $x >0$.
					\item Leaky Relu: $y_i = a_ix_i$ per $x < 0$ o $y_i = x_i$ per $x >0$.
					\item Randomized Leaky Relu: $y_{ji} = a_{ji}x_{ji}$ per $x < 0$ o $y_{ji} = x_{j}i$ per $x >0$.
				\end{itemize}
			\end{multicols}

			\paragraph{Sigmoid e Tanh}
			Questi due tipi di hidden unit riducono il tipo di non linearit\`a restringendo l'outpu ai range $[0,1]$ e $[-1,1]$.
			Si saturano lungo la maggior parte del dominio e sono fortemente sensibili unicamente quando l'input \`e pi\`u vicino a $0$.
			La saturazione rende l'apprendimento basato sul gradienet pi\`u difficile.
			\begin{multicols}{2}
				\begin{itemize}
					\item Sigmoide: $h(x) = \frac{1}{1+e^{-x}}$.
					\item Tangente iperbolica: $h(x) = \frac{e^x-e^{-x}}{e^x+e^{-x}}$.
				\end{itemize}
			\end{multicols}

		\subsubsection{Architettura}
		La decisione rispetto alla profondit\`a e larghezza di una rete neurale si basa principalmente su risultati empirici.
		Un risultato teorico da parte di Cybenko $1989$: reti a $2$-layer con output lineari con del squashing non-linearity nelle hidden-unit possono approssimare ogni funzione continua su dominio compatto ad accuratezza arbitraria.
		Questo risultato \`e valido anche per funzioni non lineari.
		Questo implica che per ogni funzione che si cerca di imparare, un grande \emph{MLP} \`e in grado di impararla.
		Nonostante questo non \`e garantici che l'algoritmo di training sia capace di imparare tale funzione.

	\subsection{Backpropagation}
	La backpropagation \`e il modo in cui la rete impara i propri pesi.
	Consiste di tre passaggi:
	\begin{multicols}{2}
		\begin{enumerate}
			\item Feedforward propagation: si accetta l'input $x$, si passa attraverso stages intermedi e si ottiene l'output.
			\item Si usa l'output computato per computare un costo scalare dipendente dalla loss function.
			\item La backpropagation permette all'informazione di fluire all'indietro dal costo per computare il gradiente.
		\end{enumerate}
	\end{multicols}
	Si utilizza il gradient descent: si necessitano le derivate degil errori per ogni peso nella rete:
	$$w^{(i)}_{jk} := w^{(i)}_{jk} - \nu\frac{\partial L}{\partial w^{(i)}_{jk}}$$
	Dai training data non si conosce cosa dovrebbero fare gli hidden layers, ma si pu\`o computare quanto velocemente l'errore cambia cambiando la loro attivit\`a: si usano le derivate dell'errore con rispetto alle sue attivit\`a.
	Ogni hidden unit pu\`o avere effetto su diverse unit\`a di output e effetti separati sull'errore che vengono combinati.
	Si pu\`o calcolare la derivata di queste unit\`a efficentemente: una volta che si ha la derivata delle attivit\`a nascoste \`e facile ottenere l'errore per i pesi che arrivano.

		\subsubsection{Operazione di feedforward}
		Dall'input all'output:
		$$\hat{y}(x;w) = f\bigl(\sum\limits_{i=1}^mw_j^{(l)}h\bigl(\cdots h\bigl(\sum\limits_{i = 1}^dw_{ij}^{(1)}x_i + w_{0j}^{(1)}\bigr)\cdots\bigr) + w_0^{(l)}\bigr)$$

		\subsubsection{Computare l'errore e train}
		L'errore della rete sul training set:
		$$L(X;w) = \sum\limits_{i=1}^N\frac{1}{2}(y_i-\hat{y}(x_i;w))^2$$
		Non ha una soluzione di forma causa, si utilizza il gradient descent.
		Si deve calcolare la derivata di $L$ su un singolo esempio.
		Si pu\`o considerare un modello lineare semplice con output $\hat{y} = \sum\limits_jw_jx_{ij}$:
		$$\frac{\partial L(X_i)}{\partial w_j} = (\hat{y}_i -y_i)x_{ij}$$

		\subsubsection{Backpropagation}
		L'unit\`a di attivazione generale in una rete multilayer:
		$$z_t = h\bigl(\sum\limits_jw_{jt}z_j\bigr)$$
		La forkard propagation calcola per ogni unit\`a $a_t = \sum\limits_j w_{jt}z_j$.
		Dove $a_t$ \`e l'input della funzione di attivazione di un'unit\`a di livello $t$.
		Ora
		$$\frac{\partial L}{\partial w_{jt}} = \frac{\partial L}{\partial a_t}z_j$$
		Sia $\partial a_t = \delta_t$ e l'unit\`a di output con attivazione lineare $\delta_t = \hat{y} -t$.
		L'unit\`a hidde $t$ che manda l'output all'unit\`a $S$:
		\begin{align*}
			\delta_t &= \sum\limits_{s\in S} = \sum\limits_{s\in S}\frac{\partial L}{\partial a_s}\frac{\partial a_s}{\partial a_t}=\\
			&= h'(a_t)\sum\limits_{s\in s} w_{ts}\delta_s
		\end{align*}

		\subsubsection{Esempio}
		Sia l'output $f(a) = a$ e l'hidden: $h(a) = tanh(a) = \frac{e^a-e^{-a}}{e^a+e^{-a}}$, ora $h'(a) = 1-h(a)^2$.
		Dato $x$, il feedforward inputs:
		\begin{multicols}{2}
			\begin{itemize}
				\item Input to hidden $a_j = \sum\limits_{i = 0}^d w_{ij}^{(1)}x_i$.
				\item Hidden input $z_j = tanh(a_j)$.
				\item Net output $\hat{y} = a = \sum\limits_{i = 0}^m w_j^{(2)} z_j$.
			\end{itemize}
		\end{multicols}
		L'errore sull'esempio $x$ \`e $L= \frac{1}{2}(y-\hat{y})^2$ e l'unit\`a di output: $\delta = \frac{\partial L}{\partial a} = y - \hat{y}$.
		Si computa $\delta$ per le unit\`a hidden:
		$$\delta_j = (1-z_y)^2w_j^{(2)}\delta$$
		Le derivate con rispetto ai pesi:
		\begin{multicols}{2}
			\begin{itemize}
				\item $\frac{\partial L}{\partial w_{ij}^{(1)}} = \delta_jx_i$.
				\item $\frac{\partial L}{\partial w_j^{(2)}} = \delta z_j$.
			\end{itemize}
		\end{multicols}
		E si aggiornano i pesi:
		\begin{multicols}{2}
			\begin{itemize}
				\item $w_j = w_j - \nu \delta z_j$.
				\item $w_{ij}^{(1)} = w_{ij}^{(1)} - \nu \delta_jx_i$.
			\end{itemize}
		\end{multicols}

			\paragraph{Output multidimensionale}
			Per l'output multidmensionale la loss sull'esempio \`e:
			$$\frac{1}{2}\sum\limits_{k = 1}^K(y_k-\hat{y}_k)^2$$
			Per ogni unit\`a di output: $\delta_k = y_k 0 \hat{y}_k$.
			Per l'unit\`a hidden $j$:
			$$\delta_j = (1-z_j)^2\sum\limits_{k=1}^Lw_{jk}^{(2)}\delta_k$$








	\subsection{Scelta di un ottimizzatore}
	Il gradient \`e il vettore delle derivate parziali con rispetto di tutte le coordinate dei pesi:
	$$\nabla_w L = \bigl[\frac{\partial L}{\partial w_1}, \cdots, \frac{\partial L}{\partial w_N}\bigr]$$
	Ogni derivata parziale misura quanto velocemente la loss cambia in una direzione.
	Quando il gradiente \`e zero, la loss non cambia in nessuna direzione.
	D\`a problemi nei saddle points e nei local minima.
	Il gradient descent trova pertanto l'insieme di parametri che rendono la loss il pi\`u piccola possibile.
	I cambi nei parametri dipendono dal gradiente della loss con rispetto dei pesi della rete.
	La backpropagation \`e il metodo per computare i gradienti.
	Si analizzano altri miglioramenti come Stochastic gradient descent.
	Il gradient descent Nelle reti neurali pu\`o essere calcolato in diversi modi.

		\subsubsection{Batch Gradient Descent (BGD)}
		In BGD i gradienti sono computati su ogni update per l'intero training con un alto costo computazionale ma garantendo una grande stabilit\`a nella stima del gradiente.
		Il learning rate $\varepsilon_k$ pu\`o cambiare nel tempo.\\
		\begin{algorithm}[H]
\DontPrintSemicolon
\SetKwComment{comment}{$\%$}{}
\SetKw{Int}{int}
\SetKw{To}{to}
\SetKw{IsNot}{is not}
\SetKw{Not}{not}
\SetKw{Return}{return}
\SetKw{Require}{return}
\SetKwData{Item}{item}
\SetKwFunction{Min}{min}
\SetKwFunction{GradientDescent}{Batch Gradient Descent at iteration K}

\caption{\protect\GradientDescent}
	\Require : Learning rate $\varepsilon_k$\;
	\Require : Initial Parameter $\theta$\;
	\While{stopping criteria not met}{
		\comment{Compute gradient estimate over $N$ examples}
		g = $+\frac{1}{N}\nabla_\theta\sum_i L(f(x^{(i)};\theta),y^{(i)}$\;
		\comment{Apply update}
		$\theta = \theta -\varepsilon_k g$\;
	}
\end{algorithm}


		\subsubsection{Stochastic gradient descent (SDG)}
		In SDG si computa il gradiente solo su un campione e non sull'intero training set in modo da ottenere performance migliori.
		Il learning rate cambia ad ogni passo, tipicamente decade linearmente.
		
\begin{algorithm}
\DontPrintSemicolon
\SetKwComment{comment}{$\%$}{}
\SetKw{Int}{int}
\SetKw{To}{to}
\SetKw{IsNot}{is not}
\SetKw{Not}{not}
\SetKw{Return}{return}
\SetKw{Require}{return}
\SetKwData{Item}{item}
\SetKwFunction{Min}{min}
\SetKwFunction{GradientDescent}{Stochastic Gradient Descent at iteration K}

\caption{\protect\GradientDescent}
	\Require : Learning rate $\varepsilon_k$\;
	\Require : Initial Parameter $\theta$\;
	\While{stopping criteria not met}{
		\comment{Compute gradient estimate over sample example ($x^{(i)}, y^{(i)}$) from training set}
		g = $+\nabla_\theta\sum_i L(f(x^{(i)};\theta),y^{(i)}$\;
		\comment{Apply update}
		$\theta = \theta -\varepsilon_k g$\;
	}
\end{algorithm}


		\subsubsection{Minibatches}
		Le minibatches risolvono il problema di SDK rispetto ai dati rumorosi.
		Il tempo di computazione per aggiornamento non dipende dal numero di esempi di training $N$, permettendo la creazione di minibatches pi\`u grandi che vengono computate parallelamente.
		Sono tipicamente di dimensione $2^n$ per le propriet\`a di calcolo della GPU.

		\subsubsection{Momento}
		Un problema di BGD e SGD \`e il fatto che minimizzano l'errore in molto tempo.
		Una soluzione diversa \`e data dalla tecnica del momento, che introduce un vettore velocit\`a $v$ di aggiornamenti, una media con decay esponenziale per i gradienti utilizzato per aggiornare i pesi.
		Introduce un vettore momento che regola il trade-off tra il gradiente allo step corrente e quelle vecchie.\\
		\begin{algorithm}
\DontPrintSemicolon
\SetKwComment{comment}{$\%$}{}
\SetKw{Int}{int}
\SetKw{To}{to}
\SetKw{IsNot}{is not}
\SetKw{Not}{not}
\SetKw{Return}{return}
\SetKw{Require}{return}
\SetKwData{Item}{item}
\SetKwFunction{Min}{min}
\SetKwFunction{GradientDescent}{Stochastic Gradient Descent with momentum}

\caption{\protect\GradientDescent}
	\Require : Learning rate $\varepsilon_k$\;
	\Require : Momentum parameter $\alpha$\;
	\Require : Initial Parameter $\theta$\;
	\Require : Initial velocity $v$\;
	\While{stopping criteria not met}{
		\comment{Compute gradient estimate over sample example ($x^{(i)}, y^{(i)}$) from training set}
		g = $+\nabla_\theta\sum_i L(f(x^{(i)};\theta),y^{(i)}$\;
		\comment{Compute velocity update}
		$v = \alpha v - \varepsilon_k g$\;
		\comment{Apply update}
		$\theta = \theta -\varepsilon_k g$\;
	}
\end{algorithm}


		\subsubsection{Adaptive learning rate method}
		Alcune volte \`e bene usare un learning rate diverso per ogni peso.
		Un metodo che lo implementa \`e Adagrad o adaptive gradient optimizer.
		Questo fa downscale di un parametro del modello della radice della somma dei quadrati dei valori storici, in questo modo parametri con una grande derivata parziale hanno learning rate che si abbassano rapidamente
		Si adatta al learning rate dei parametri svolgendo aggiornamenti minori associati con feature che pi\`u frequenti e grandi per feature meno frequenti.
		Utile per gestire dati sparsi.\\
		\begin{algorithm}[H]
\DontPrintSemicolon
\SetKwComment{comment}{$\%$}{}
\SetKw{Int}{int}
\SetKw{To}{to}
\SetKw{IsNot}{is not}
\SetKw{Not}{not}
\SetKw{Return}{return}
\SetKw{Require}{return}
\SetKwData{Item}{item}
\SetKwFunction{Min}{min}
\SetKwFunction{GradientDescent}{AdaGrad}

\caption{\protect\GradientDescent}
	\Require : Learning rate $\varepsilon_k$\;
	\Require : Initial Parameter $\theta,\delta$\;
	r = 0\;
	\While{stopping criteria not met}{
		\comment{Compute gradient estimate over sample example ($x^{(i)}, y^{(i)}$) from training set}
		$\hat{g} = +\nabla_\theta\sum_i L(f(x^{(i)};\theta),y^{(i)})$\;
		\comment{Compute velocity update}
		r = r + $f\circ \hat{g}$
		\comment{Compute update}
		$\Delta\theta = -\frac{\varepsilon_k}{\delta+\sqrt{r}}\circ i\hat{g}$\;
		$\theta = \theta +\Delta\theta$\;
	}
\end{algorithm}


\section{Convolutional Neural Netowrks (CNN)}
Un tipo di FFNN sono le CNN.

	\subsection{Convolution}
	La convolution avviene quando un filtro, una piccola matrice, viene applicata a una matrice pi\`u grande, poi delle operazioni vengono svolte con il filtro ottenendo una nuova matrice.
	L'idea della convolution nasce dal fatto che l'occhio umano processa le immagini in livelli, in cui ogni livello \`e specializzato in certe features e pi\`u profondo il layer, pi\`u complesse le target features.
	La corteccia dell'occhio contiene un complesso ordinamento\section{Convolutional Neural Netowrks (CNN)}
	Un tipo di FFNN sono le CNN.
	Queste sono molto utili quando l'obiettivo non \`e la classificazione: ottengono grandi risultati in region extraction, feature detection, semantic segmentation e structured regression.

		\subsection{Convolution}
		La convolution avviene quando un filtro, una piccola matrice, viene applicata a una matrice pi\`u grande, poi delle operazioni vengono svolte con il filtro ottenendo una nuova matrice.
		L'idea della convolution nasce dal fatto che l'occhio umano processa le immagini in livelli, in cui ogni livello \`e specializzato in certe features e pi\`u profondo il layer, pi\`u complesse le target features.
		La corteccia dell'occhio contiene un complesso ordinamento di cellule, sensibili a piccole regioni del campo visivo detto receptive field.
		Queste cellule agiscono come filtri locali sull'input space e sono ben definite per sfruttare la correlazione locale in immagini naturali.
		Le cellule semplici rispondono a pattern specifici simili a separazioni nel receptive field, mentre le complesse hanno un receptive field maggiori e sono invarianti locali della posizione del pattern.
		Il filtraggio che si volge nelle CNN tenta di emulare questi calcoli specializzati.
		La convolution \`e un operazioni di filtraggio general purpose per le immagini.
		Una matrice kernel viene applicata a un'immagine determinando il calore di un pixel centrale aggiungevo ad esso i valori pesati dei suoi vicini.
		L'output \`e una nuova immagine filtrata.

		\subsection{Definizione di una CNN}
		Si definisce una CNN come una FFNN con struttura di connessione specializzata, dove layer di basso livello estraggono features locali e quelli di alto livello estraggono pattern globali.
		Ci sono tre tipi di layer nelle CNN:
		\begin{itemize}
			\item Convolution: questa operazione viene svolta shiftando la matrice di kernel sul dato originale utilizzandola come filtro.
			\item Non-linearity: utilizza una funzione non lineare di attivazione come filtro.
			\item Pooling: l'idea dietro il pooling \`e quella di ridurre la dimensione della feature map.
		\end{itemize}

		\subsection{Operazione di convolution}
		Un layer convolutional consiste di una serie di filtri.
		Ogni filtro copre una piccola parte dei dati di input o receptive field.
		Ogni filtro \`e convolved attraverso la dimensione degli input data, producendo una mappa di feature multi-dimensionale.
		La rete impara filtri specifici che si attivano quando trovano feature specifiche a una posizione particolare dell'input.

		\subsection{Non linearity}
		$$y_{i,j} = f(a_{i,j})\ dove\ f(a)= sigmoid(a)$$

		\subsection{Pooling}
		Il pooling riduce la dimensione dell'input attraverso filtri.
		L'output avr\`a la dimensione dei filtri.
